\documentclass[xcolor=dvipsnames]{beamer} 
\usepackage[polish]{babel}
\usepackage[utf8]{inputenc}
\usepackage{polski}
\usecolortheme[named=Brown]{structure} 
\useoutertheme{infolines} 
\usetheme[height=7mm]{Rochester}
\author{Grzegorz Wieczorek} 
\title[Systemy plików]{Systemy plików }
\institute[StuKoLi]{
  Szkoła Główna Gospodarstwa Wiejskiego\\
  Wydział Zastosowań Informatyki i Matematyki\\
  Informatyka\\[1ex]
}
\begin{document}
\begin{frame}[plain]
  \titlepage
\end{frame}
\begin{frame}
  \tableofcontents
\end{frame}
\section{Przegląd systemów}
\begin{frame}{ext2}
Partycje o wielkości do 16 TB, pojedyncza pliki o wielkości 2 TB. Nazwy plików mogą mieć do 255 znaków długości.
\end{frame}
\begin{frame}{ext3}
Partycje o wielkości do 32 TB. Jest kompatybilny wstecz. Mogą z niego kozystać programy napisane dla ext2.
\end{frame}
\begin{frame}{ext4}
Następca ext3. Partycje o wielkości do 1 eksbibajta. Wielkość pojedynczego pliku do 16 tebibajtów. Włączenie niektórych opcji pozbawia kompatybilności z systemem ext3.
\end{frame}
\section{Działania na partycjach}
\begin{frame}{Działania na partycjach}
\begin{block}{Tworzenie systemu plików}
mkfs -t typ /dev/sdX
\newline
mkfs.ext3 /dev/sdX
\end{block}
\pause
\begin{block}{Montowanie}
mount -t typ /dev/sdX /mnt/katalog
\newline
mount -t iso9660 -o ro /dev/cdrom /cdrom0
\end{block}
\pause
\begin{block}{Label}
e2label /dev/sdX nazwa
\newline
mount LABEL=nazwa /mnt/katalog
\newline
findfs LABEL=nazwa
\end{block}
\end{frame}
\begin{frame}{Działania na partycjach}
\begin{block}{Odmontowywanie}
umount /dev/sdX
\newline
umount /mnt/katalog
\end{block}
\pause
\begin{block}{Sprawdzanie systemu plików}
e2fsck -f /dev/sdX
\end{block}
\pause
\begin{block}{Zmiana rozmiaru - zwiększanie}
resize2fs /dev/sdX
\end{block}
\pause
\begin{block}{Zmiana rozmiaru - zmniejszanie}
resize2fs /dev/sdX 10m
\newline
lvreduce -L 10 /dev/sdX
\end{block}
\end{frame}
\begin{frame}{Automatyczne montowanie przy starcie systemu}
Za automatyczne montowanie systemów plików odpowiada plik /etc/fstab
\newline
Struktura tego pliku wygląda następująco:
\newline
/dev/sdX	/	ext3	defaults 1 1
\newline
LABEL=home	/home	ext3	defaults 1 1
\newline
\pause
\begin{block}{Montowanie wszystkiego z pliku fstab}
mount -a
\end{block}
\end{frame}
\section{ACL}
\begin{frame}{ACL}
\begin{block}{Wyświetlanie danych o pliku}
getfacl plik
\end{block}
\pause
\begin{block}{Czy partycja obsługuje acl}
tune2fs -l /dev/sdX | grep acl
\newline
mount -o remount, acl /home
\newline
tune2fs /dev/sdX -o acl
\end{block}
\pause
\begin{block}{Ustawianie ACL}
setfacl -m u:user2:rw plik
\end{block}
\pause
\begin{block}{Kasowanie ACL}
setfacl -x u:user2:rw plik
\end{block}
\end{frame}
\section{quota}
\begin{frame}{quota}
\begin{block}{Sprawdzenie czy jest pakiet}
rpm -q quota
\end{block}
\pause
\begin{block}{Montowanie z obsługą quoty}
mount -o remount, acl, usrquota, grpquota /home
\newline
mount -v | grep home
\end{block}
\pause
\begin{block}{Sprawdzenie quoty}
quotacheck -cgmuv /home
\end{block}
\pause
\begin{block}{Aktywacja quoty}
quotaon -ugv /home
\end{block}
\end{frame}
\begin{frame}{quota}
\begin{block}{Ustawianie quoty dla usera}
setquota -u user 30000 35000 0 0 /home
\end{block}
\pause
\begin{block}{Raportowanie quoty (user)}
repquota -a
\end{block}
\pause
\begin{block}{Raportowanie quoty (grupa)}
repquota -ag
\end{block}
\pause
\begin{block}{Raportowanie quoty dla udziału}
repquota -v /home
\end{block}
\pause
\begin{block}{Wyłączanie quoty}
quotaoff -ugv /home
\end{block}
\end{frame}
\section{Swap}
\begin{frame}{Swap}
\begin{block}{Informacje o pamięci}
cat /proc/meminfo
\end{block}
\pause
\begin{block}{Tworzenie partycji swap}
mkswap /dev/sdX
\newline
swapon /dev/sdX
\end{block}
\pause
\begin{block}{Statystyki swap}
free -m
\newline
swapon -s
\newline
cat /proc/swaps
\newline
vmstat
\end{block}
\end{frame}
\begin{frame}{Koniec}
\begin{center}
Dziękuję za uwagę.
\end{center}
\end{frame}
\end{document}